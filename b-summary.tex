\noindent {\bf \large Overview:}\\

\noindent We approach materials discovery via three core thrusts: advanced Bayesian optimization, novel discovery algorithms, and autonomous experimentation. Materials optimization tasks are often hierarchical, constrained, high-dimensional, multi-objective, noisy, sparse, and multi-fidelity and contain both numerical and categorical variables. One thrust of the project involves leveraging state-of-the-art Bayesian optimization algorithms (e.g., expected hypervolume improvement for multi-objective optimization) to accelerate the discovery of new materials. The topics to be explored are high-dimensionality, linear and nonlinear constraints, multiple fidelities, and multiple objectives. For the second thrust of novel discovery algorithms, we will explore two topics: density obtained via density-aware embeddings as a proxy for novelty and crystal structure generative modeling to generate realistic and synthetically accessible crystal structures. Finally, the third thrust, autonomous experimentation, involves the development of teaching and prototyping demos for light-, liquid-, and solid-handling tasks with progressively increasing difficulty. The culminating demo will be an autonomous robotic chocolatier demo that will serve as the skeleton for a future autonomous alloy discovery laboratory and incorporate advanced optimization tutorials. \\

\noindent {\bf \large Intellectual Merit:}\\

\noindent Bayesian optimization is relevant for materials discovery because it allows for more efficient probing of a search space in search of high-performing candidates. This is achieved by elegantly navigating the trade-off between exploiting known, high-performing regions and exploring uncertain regions. Advanced Bayesian optimization topics are necessary for the complex situations often a part of industry-relevant materials discovery, such as the presence of dozens of tunable variables (high-dimensional), compositional and NChooseK constraints (linear and nonlinear constraints), the existence of both simulated and experimental data (multi-fidelity), and the presence of multiple competing performance  targets (multi-objective). Moving past screening for high-performing materials, emphasizing novelty can help unlock new avenues for discovery and drive the transformation of a field. For example, the discovery of new classes of superconductors can reveal new mechanisms and curb the stagnation that results from simply doing local optimization with small perturbations to well-known systems.  \\

\noindent {\bf \large Broader Impacts: }\\

\noindent Advanced Bayesian optimization and novel materials discovery algorithms have the potential to accelerate the discovery of new materials and new knowledge from the computational side; by pairing these with automation hardware to synthesize and characterize materials in a closed loop, the benefits can be amplified even further. This could lead to the development of new technologies and the creation of new markets and industries, making a significant impact on society. However, the barrier to entry for autonomous experimentation is often high as it requires knowledge from diverse disciplines and a large amount of capital for industry-relevant materials discovery tasks. "Hello, World!" examples of autonomous scientific discovery are necessary to train the next generation of scientists. These simple demonstrations serve as a starting point for more complex systems. By providing a platform for students to learn and engage with autonomous experimentation, we can increase the number of people who can contribute to the field and further advance materials discovery. Code and data generated during the studies will also be made available, improving the accessibility and reusability of the methods described. %Tell reviewers about other outcomes of your project beyond the science. Ideas include ancillary products, outreach, and more.