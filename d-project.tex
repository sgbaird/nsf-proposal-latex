%%%%%%%%%%%%%%%%%%%%%%%%%%%%%%%%%%%%%%%%%%%%%%%
% These are the general sections to include.  %
%                                             %
% You can alter some names, but follow the    %
% suggestions in the NSF guidelines.          %
%                                             %
% If spacing is tight, play with negative     %
% vspaces w/in the text to reduce whitespace. %
%%%%%%%%%%%%%%%%%%%%%%%%%%%%%%%%%%%%%%%%%%%%%%%

%%%%%%%%%%%%%%%%%%%%%%%%%%%%%%
% Section 1: Introduction    %
%%%%%%%%%%%%%%%%%%%%%%%%%%%%%%
\section{Introduction}
\label{intro}

Materials optimization tasks are often hierarchical, constrained, high-dimensional, multi-objective, noisy, sparse, and multi-fidelity and contain both numerical and categorical variables. As a result, traditional optimization methods may not effectively find the optimal solution. Advanced optimization techniques, such as Bayesian optimization and generative modeling, can be helpful in these types of materials optimization tasks because they can handle complex search spaces and incorporate domain knowledge. These methods can also handle constraints, multiple objectives, and noisy or sparse data, making them well-suited for materials optimization tasks. f

%%%%%%%%%%%%%%%%%%%%%%%%%%%%%%
% Section 2: Overview        %
%%%%%%%%%%%%%%%%%%%%%%%%%%%%%%
\section{Background}


\subsection{Bayesian Optimization}

% pull from self-driving-lab-demo tutorial notebooks

\subsubsection{A sub-subsection}


\subsection{Novel Discovery}

% pull from DiSCoVeR paper

\subsection{Self-driving Laboratories}

% pull from matter paper


%%%%%%%%%%%%%%%%%%%%%%%%%%%%%%
% Section 3: Research Plan   %
%%%%%%%%%%%%%%%%%%%%%%%%%%%%%%
\section{Research Methodology}


\subsection{A component of your plan}

\setlength\intextsep{0pt}
\begin{wrapfigure}[20]{R}{2.4in}
\vspace{-5pt}
\centering
\includegraphics[width=2.4in]{figures/sample1}
\caption{A sample figure that is wrapped by text.}
\label{fig1}
\end{wrapfigure}


\begin{wraptable}[15]{r}[0.01in]{4.5in}
\label{table1}
\caption{A sample table wrapped by text.}
\begin{center}
\vspace{-10pt}
\scriptsize
\begin{tabular}{  c  c  c  }
\hline
\hline
Stability & $M_u$ & $M_{\tau}$ \\ 
\hline\hline\\
Neutral & $\cfrac{z}{z_\Delta} - \cfrac{\ln(z/z_o)}{\ln(z_\Delta/z_o)}$ & $\cfrac{z}{z_\Delta} - \cfrac{1}{\ln(z_\Delta/z_o)}$\\\\
\hline \\
Stable & $\left(1 - \cfrac{\Psi}{2}\right)\cfrac{z}{z_\Delta} - \left(1 - \cfrac{\Psi_\Delta}{2}\right)
\left(\cfrac{\ln(z/z_o)-\Psi}{\ln(z_\Delta/z_o)- \Psi_\Delta}\right)$ & $\cfrac{z}{z_\Delta} - \cfrac{\left(1 - \cfrac{\Psi_\Delta}{2}\right)}{\ln(z_\Delta/z_o) - \Psi_\Delta}$\\\\
\hline \\
Unstable & $\cfrac{4}{3}\left[\left(\cfrac{1-x^3}{1-x_{\Delta}^{4}}\right) -  \left(\cfrac{1-x_{\Delta}^3}{1 - x_{\Delta}^{4}}\right)\left(\cfrac{\ln(z/z_o)-\Psi}{\ln(z_\Delta/z_o)- \Psi_\Delta}\right)\right]$ & $\cfrac{z}{z_\Delta} - \cfrac{\cfrac{4}{3}\left(\cfrac{1-x_{\Delta}^3}{1 - x_{\Delta}^{4}}\right)}{\ln(z_\Delta/z_o) - \Psi_\Delta}$\\\\
\hline
\hline
\end{tabular}
\end{center}
\end{wraptable}



% I found it useful to include a summary of the proposed work
% given in each subsection to help out reviewers.
\subsubsection{Specific tasks for this research component}
\begin{itemize}
\setlength\itemsep{0em}
\item Do a thing and blow your mind
\item Question your life choices
\item Drink coffee
\end{itemize}

\begin{center}
\begin{minipage}{.3\textwidth}
\begin{equation}
 \bar u = \bar u_{ll} + u_* \beta M_{u} \label{new_u}
\end{equation}
\end{minipage}
\begin{minipage}{.36\linewidth}
\begin{equation}
  \tau_{xz} = u_* u_{*ll} + \kappa u_*^2 \beta M_{\tau} \label{new_tau} \mbox{ ,}
\end{equation}
\end{minipage}
~\\Sample equations that consume minimal space.
\end{center}



%%%%%%%%%%%%%%%%%%%%%%%%%%%%%%
% Section 4: Management Plan %
%%%%%%%%%%%%%%%%%%%%%%%%%%%%%%
\section{Time Line and Management Plan}

\begin{table}[H]
\label{table1}
\renewcommand{\arraystretch}{0}
\caption{Project schedule.  PIs are Person One (P1), Person Two (P2), graduate student is GS, and the undergraduate student is US.\ Time frame gives the year each activity will occur.}
\scriptsize
\begin{tabularx}{\textwidth}{Y c c }
\hline
\hline
\textbf{Research Activity} & \textbf{Personnel} & \textbf{Time Frame}\\
\hline
Perform a task that sounds impressive & P2, US & Y1 \T\\
Perform another super-amazing task & P1, US & Y1 \T\\
Perform something else that may not be as sexy as the other things & P2, GS & Y1 \T\\
Wonder why you are such a terrible programmer & P1, US & Y1 \T\\
Analyze the results and stuff & P1, P2, SS & Y1,Y2 \T\\
Take the day off and grill some meat & P1, P2, SS & Y1,Y2 \T\\
Present findings at scientific meetings and publish results in peer-reviewed journals & P1, P2, US, GS & Y1, Y2, Y3\T\B\\
\hline
\hline
\end{tabularx}
\end{table}

%%%%%%%%%%%%%%%%%%%%%%%%%%%%%%
% Section 5: Science Merit   %
%%%%%%%%%%%%%%%%%%%%%%%%%%%%%%
\section{Scientific Merit}



%%%%%%%%%%%%%%%%%%%%%%%%%%%%%%
% Section 6: Impact/Outreach %
%%%%%%%%%%%%%%%%%%%%%%%%%%%%%%

\section{Broader Impacts}
\label{broadimpacts}
\vspace*{-8pt}

This project will have direct impacts on research and education through access to simulation data products, student training, and K-12 outreach.  

\vspace{4pt}
\noindent \underline{\textit{Data Access}}: Maybe write about you will make data available.

\vspace{4pt}
\noindent \underline{\textit{Student Training}}: Write about how you will train students.

\vspace{4pt}
\noindent \underline{\textit{Some Other Outreach}}: Write about more outreach.

\vspace{4pt}
\noindent \underline{\textit{Dissemination}}: Write about how you will disseminate results (i.e., journal articles, workshops, etc).

%%%%%%%%%%%%%%%%%%%%%%%%%%%%%%
% Section 7: Prior NSF Work  %
%%%%%%%%%%%%%%%%%%%%%%%%%%%%%%
\section{Results from Prior NSF Support}

\noindent \emph{\underline{Person One}}: No NSF support in the past five years \newline

\noindent The most relevant prior NSF award to the proposed project for \underline{Person Two} (Co-PI) is: (a) NSF PDM \#\#\#\#\#\#\#, \$000,000, MM/DD/YY to MM/DD/YY; (b) Title: Super Cool Project That Got Funded; (c) Accomplishments related to the {\bf intellectual merit} of this research project include something something. The {\bf broader impacts} include outreach at many levels. Something Something. To date, the grant has funded one post-doc and 1000 graduate students. The project has also involved 500 undergraduate students. (d) To date this project has resulted in 100 conference presentations, one million journal publications (cite them) with one under review (cite it) and two in preparation with well-developed drafts.

\nocite*